% This text is proprietary.
% It's a part of presentation made by myself.
% It may not used commercial.
% The noncommercial use such as private and study is free
% Sep. 2005 
% Author: Sascha Frank 
% University Freiburg 
% www.informatik.uni-freiburg.de/~frank/


\documentclass{beamer}
\begin{document}
\title{Simple Beamer Class}   
\author{Sascha Frank} 
\date{\today} 

\frame{\titlepage} 

\frame{\frametitle{Table of contents}\tableofcontents} 


\section{Variational Inference } 
\frame{\frametitle{Overview of Variational Inference}
  \begin{itemize}
    \item Turns a complicated inference problems into an optimization problem
    \item Minimizes the KL divergence from the variational distribution to the posterior
      distribution
    \item EM is a special case
  \end{itemize}
}

\frame{\frametitle{Evidence Lower Bound (ELBO)}
  \begin{itemize}
    \item Key insight is the use of Jensen's inequality
    \item We choose $q$ from some family of distributions $\mathcal{Q}$
  \end{itemize}

  \begin{eqnarray*} %% Do avoid eqnarray if possible.
    \log(p(x) & \log \int p(x,z,\beta) dz d\beta \\
    & = & \log \int p(x,z,\beta) \frac{q(z,\beta)}{q(z,\beta)} dz d\beta \\
    & = & \log\left(\mathbb{E}_q \left[ \frac{p(x,z,\beta)}{q(z,\beta)} \right] \right) \\
    & \geq & \mathbb{E}_q \left[ \log p(x,z,\beta)\right] - \mathbb{E}_q \left[ q(z,\beta) \right] \\
    & = & \mathcal{L}(q)
  \end{eqnarray*}
}

\frame{\frametitle{Why does maximizes the ELBO minimize the KL-Divergence?} 
  \begin{eqnarray*} %% Do avoid eqnarray if possible.
    \text{KL}(q(z,\beta)||p(z,\beta|x) &=& \mathbb{E}_q \left[ \log(q(z,\beta) \right] - \mathbb{E}_q 
\left[ \log p(z,\beta|x) \right] \\
    & = & \mathbb{E}_q \left[ \log(q(z,\beta) \right] - \mathbb{E}_q 
\left[ \log p(z,\beta,x) \right] + \log p(x) \\
    & = & -\mathcal{L}(q) + const.
    \end{eqnarray*}
    
    \begin{itemize}
      \item Maximizing $\mathcal{L}(q)$ is just minimizing $-\mathcal{L}(q)$
    \end{itemize}

}

\frame{\frametitle{How is EM a special case?}
  \begin{itemize}
    \item How do we choose $\mathcal{Q}$?
    \item We want it to be easily computable!
    \item What if $p(z,\beta|x) \in \mathcal{Q}$? 
    \item Minimizing the KL divergence is trivial! Set $q(z,\beta) = p(z,beta|x)$
  \end{itemize}
  

}

\frame{\frametitle{Variational + $x,\; \forall x$}
  \begin{itemize}
    \item Clearly we have some choices...and they have names!
  \end{itemize}

}


\section{Section no. 2} 
\subsection{Lists I}
\frame{\frametitle{unnumbered lists}
\begin{itemize}
\item Introduction to  \LaTeX  
\item Course 2 
\item Termpapers and presentations with \LaTeX 
\item Beamer class
\end{itemize} 
}

\frame{\frametitle{lists with pause}
\begin{itemize}
\item Introduction to  \LaTeX \pause 
\item Course 2 \pause 
\item Termpapers and presentations with \LaTeX \pause 
\item Beamer class
\end{itemize} 
}

\subsection{Lists II}
\frame{\frametitle{numbered lists}
\begin{enumerate}
\item Introduction to  \LaTeX  
\item Course 2 
\item Termpapers and presentations with \LaTeX 
\item Beamer class
\end{enumerate}
}
\frame{\frametitle{numbered lists with pause}
\begin{enumerate}
\item Introduction to  \LaTeX \pause 
\item Course 2 \pause 
\item Termpapers and presentations with \LaTeX \pause 
\item Beamer class
\end{enumerate}
}

\section{Section no.3} 
\subsection{Tables}
\frame{\frametitle{Tables}
\begin{tabular}{|c|c|c|}
\hline
\textbf{Date} & \textbf{Instructor} & \textbf{Title} \\
\hline
WS 04/05 & Sascha Frank & First steps with  \LaTeX  \\
\hline
SS 05 & Sascha Frank & \LaTeX \ Course serial \\
\hline
\end{tabular}}


\frame{\frametitle{Tables with pause}
\begin{tabular}{c c c}
A & B & C \\ 
\pause 
1 & 2 & 3 \\  
\pause 
A & B & C \\ 
\end{tabular} }


\section{Section no. 4}
\subsection{blocs}
\frame{\frametitle{blocs}

\begin{block}{title of the bloc}
bloc text
\end{block}

\begin{exampleblock}{title of the bloc}
bloc text
\end{exampleblock}


\begin{alertblock}{title of the bloc}
bloc text
\end{alertblock}
}
\end{document}

